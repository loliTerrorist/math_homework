\documentclass[A4paper,12pt]{article}
\usepackage[utf8]{inputenc}
\usepackage[margin=2cm]{geometry}
\usepackage{amsmath}
\usepackage{amssymb}
\usepackage{enumerate}
\usepackage{amsthm}
\usepackage{graphicx}
\usepackage{comment}
\usepackage{xcolor}

\theoremstyle{definition}
\newtheorem{definition}{Definition}[section]
\newtheorem{theorem}{Theorem}[section]
%\theoremstyle{bfremark}
\newtheorem{remark}{Remark}[section]
\newtheorem{exmp}{Example}[section]
\newtheorem*{corollary}{Corollary}
\newtheorem{proposition}{Proposition}[section]
\newtheorem{lemma}{Lemma}[section]
\newtheorem{claim}{Claim}

\setlength{\parindent}{0cm}
\numberwithin{equation}{section}

\title{Abstract Algebra I HW1}
\author{Chengyu Hsieh, B13201053}
\date{}
\begin{document}
\maketitle
\begin{enumerate}[1)]
    \item
            From now on, for simplicity's sake, we let $f_{i}f_{j}$ denote $f_{i}(f_{j}(x))$.
        \begin{enumerate}[(a)]
            \item
                By Calculation we find that $f_{2}, f_{3}, f_{6}$ satisfy the required property.
            \item
                Note that $(12)(12), (13)(13), (23)(23)$ all equal to (1). Also see that $f_{6} = f_{2}f_{5}$, $f_{3} = f_{2}f_{5}f_{5}$, $f_{4} = f_{5}f_{5}$. Furthermore, $(13)=(12)(123)(123), (23)=(12)(123), (132)=(123)(123)$.
                Our correspondence should have the property that if $f_{i}$ corresponds to $x$, $f_{j}$ corresponds to $y$, then $f_{i}f_{j}$ corresponds to $xy$.
                By this rule, we let $f_{2}$ correspond to (12), and $f_{5}$ correspond to (123). We may check via calculation that this satisfy the mentioned rule.
                Hence one correspondence is for $f_{2}$ to correspond to $(12)$, $f_{6}$ to correspond to $(13)$ and $f_{3}$ to correspond to $(23)$. 
            \item
                $f_{2}f_{6}  = f_{4}$, $f_{6}f_{2} = f_{5}$, $f_{2}f_{3} = f_{5}$, $f_{3}f_{2} = f_{4}$, $f_{3}f_{6} = f_{5}$, $f_{6}f_{3} = f_{4}$. Hence for two distinct $f_{i}, f_{j}$ obtained in (a), we have $f_{i} \neq f_{j}$.
                In (b) we already have that if $f_{i}$ corresponds to $x$, $f_{j}$ corresponds to $y$, then $f_{i}f_{j}$ corresponds to $xy$.
                It follows that for $x, y \in \{(12), (23), (13)\}, x \neq y$, we have $xy \neq yx$.


            \item
                $f_{5}$ corresponds to $(123)$, and $f_{4}$ corresponds to $(132)$.

        \end{enumerate}
    \item 
        We assume 3) is already proven. By Claim 1, we will list the elements that are coprime with $n$. Since the calculations are too trivial and repetitive, we do not show it here.
        \begin{enumerate}[(a)]
            \item
                $
                \{1, 2, 3, 4\}
                $
            \item
                $ 
                \{1, 5\}
                $
            \item 
                $ 
                \{1, 3, 5, 7\}
                $
            \item 
                $ 
                \{1, 7, 13\}
                $
            \item 
                $ 
                \{1,2,3,4,5,6,7,8,9,10,11,12\}
                $
        \end{enumerate}
    \item 
        \begin{enumerate}[(a)]
            \item
                \textbf{Necessity:}
                Assume that $x \in (\mathbb{Z}/n\mathbb{Z})^\times$. Then by definition, there exists $y \in \mathbb{Z}/n\mathbb{Z}$ such that $xy \equiv 1 \mod{n}$. Hence $xy = \beta n+1$ for some $\beta \in \mathbb{Z}$. If we let $a = y$ and $b = -\beta$ then $ax+bn=1$.\\\\
                \textbf{Sufficiency:}
                Assume that there exists integers $a, b$ such that $ax+bn = 1$. Let $a = \alpha + \beta n,\, 0 \leq \alpha \leq n-1,\, \alpha,\beta \in \mathbb{Z}$.\\
                Thus $0 \le \alpha \le n-1$ and is in $\mathbb{Z}/n\mathbb{Z}$.
                \begin{align*}
                    &\alpha x + (b+\beta)n = 1\\
                    &\Rightarrow
                    x\alpha = -(\beta+b)n + 1\\
                    &\Rightarrow
                    x\alpha \equiv 1 \mod{n}\\
                \end{align*}
                By our definition we have $x \in (\mathbb{Z}/n\mathbb{Z})^\times$.
            \item
                \begin{claim}
                    Let $a, b \in \mathbb{Z},\, a,b>0$. If $\exists c,d\in \mathbb{Z} \text{ such that } ca+db=1,\, \text{then} (a,b)=1.$
                    \proof
                    Suppose $(a, b) = k>1$. Let $a = k\alpha,\, b = k\beta$. Let $c, d$ be integers such that $ca+db=1$. Subsitute in $k\alpha \text{ and } k\beta$ we obtain
            `        
                    \begin{align*}
                        &k(c\alpha+d\beta) = 1
                    \end{align*}
                    Since $k \text{ and }c\alpha+d\beta$ are integers, we have $(k=1 \land c\alpha+d\beta = 1) \lor (k = -1 \land c\alpha+d\beta = -1)$. But we assumed that $k>1$ and so we arrived at an contradiction.
                    Hence $(a,b) = 1$.
                \end{claim}
                \begin{claim}
                    Let $a, b \in \mathbb{Z},\, a,b>0$. If $(a,b)=m$, then exists $c, d \in \mathbb{Z} \text{ such that } ca+db = m$.
                    \proof
                    First note that if $d|a \land d|b$, then $d|m$. Now let $M = \{ca+db|c, d \in \mathbb{Z}\}$. $1a+0b = a > 0$, so $M$ has positive integers. Let $M^{+} = \{ca+db>0|c, d \in \mathbb{Z}\}$, which is non empty. Clearly $\min{M^{+}}$ exist. Let $m' = \min{M^{+}}$. Then $m' = c'a+d'b$ for some $c', d' \in \mathbb{Z}$. Now for any $x = ca+db \in M$, let $x = m'q+r$ with $0 \leq r <m'$.
                \begin{align*}
                    &r = x-m'q = (c-c'q)a + (d-d'b)b \in M
                \end{align*}
                Since we have $0 \leq r < m'$, we have $r=0$. (Otherwise $m' \neq \min(M^{+}))$.)
                Therefore $m'|x \forall x \in M$. Note that $a, b\in M \Rightarrow m'|a \land m'|b$. Also, for any $d \text{ such that } d|a \land d|b$ we have $d|c'a+d'b$, so $d|m'$. Hence $m' = (a,b) = m$.
                We conclude that $\exists c, d \in \mathbb{Z} \text{ such that } ca+db = m$.
                \end{claim}
                \textbf{Sufficiency:}\\
                Suppose $n$ is prime. 
                Then for any $x \neq 0 \in \mathbb{Z}/n\mathbb{Z}$ we have $(x,n) = 1$. 
                Hence by Claim 2, there exists $c, d \in \mathbb{Z} \text{ such that } cx+dn=1$. 
                Then by (a), we have $x \in (\mathbb{Z}/n\mathbb{Z})^{\times}$. 
                Since $(\mathbb{Z}/n\mathbb{Z})\backslash \{0\}$ has $n-1$ elements and every element of $(\mathbb{Z}/n\mathbb{Z})\backslash\{0\}$ is in $(\mathbb{Z}/n\mathbb{Z})^{\times}$, we conclude that $(\mathbb{Z}/n\mathbb{Z})^{\times}$ has $n-1$ elements.
                \\
                \textbf{Necessity:}\\
                Suppose $(\mathbb{Z}/n\mathbb{Z})^{\times}$ has $n-1$ elements. 
                Clearly, $(\mathbb{Z}/n\mathbb{Z})^{\times} = \{1, 2, ..., n-1\}$.
                By Claim 1, we have $(x, n)=1$ for every $x$ such that $1 \leq x \leq n-1$. 
                Therefore $n$ is prime.
        \end{enumerate}
    \item
        \begin{enumerate}[(a)]
            \item
                Suppose $e, e'$ are identity elements of $G$. 
                Since $e * x = x \forall x \in G$,
                $$e*e' = e'$$
                Since $x*e' = x \forall x \in G$,
                $$e*e' = e$$
                Therefore $e = e'$.
                We conclude that the identity element is unique.\\\\
                Now for an element $x \in G$, suppose that $b, c$ are both the inverse of $x$.
                $$b = b*e = b*(x*c) = (b*x)*c = e*c = c$$
                Hence the inverse of an element is unique.
            \item
                In the following discussion we let $e$ denote the identity element of G. Also, the fact that $ex = xe$ for $x\in G$ is very clear and will not be checked from now on.\\
                
                Case $|G|=1$:\\
                Let $G = \{e\}$. Trivial.\\
                Case $|G|=2$:\\
                Let $G = \{e, a\}$. Trivial.\\
                The cases from now on are not so trivial, so we introduce and prove the following claims:\\
                \begin{claim}
                    For a fixed $x \in G$ we have $xy_{1} \neq xy_{2}$ for $y_{1}, y_{2} \in G,\, y_{1} \neq y_{2}.$
                    \proof
                    Assume otherwise. Then $y_{1} =x^{-1}xy_{1} = x^{-1}xy_{2}=y_{2}$, contradicting our assumption.
                \end{claim}
                \begin{claim}
                    Let $x, y \in G, x, y \neq e$. Then $xy\neq x \text{ and } xy\neq y$.
                    \proof
                    Assume $xy = x$. Then $y = x^{-1}xy = x^{-1}x = e$, giving a contradiction. Hence $xy\neq x$. Assuming $xy=y$ gives a similar contradiction. 
                \end{claim}
                Case $|G|=3$:\\
                Let $G = \{e, a, b\}$.
                By Claim 3, $a^{2} \neq a$, so $a^{2} = a \text{ or } b$. \\
                If $a^{2} = e$, by Claim 3 we have $ab = b$ which contradict Claim 4. \\
                Hence $a^{2} = b$. By Claim 3 we have $ba = e = ab$. By Claim 3 again we have $b^{2} = a$.
                Hence $ab = ba$.\\
                Case $|G|=4$:\\
                By Claim 3, $a^{2} \neq a$.
                \\If $a^{2} = e$:\\
                By Claim 3, $ab, ba \neq a, e$. By Claim 4, $ab, ba \neq b$. Hence $ab = c = ba$. By Claim 3, $ac = b = ca$.
                Note that $b^{2} \neq c$ by Claim 3.
                \\If $b^{2} = e$:\\
                By Claim 3, $bc = a = cb$, $c^{2} = e$.
                In this case, $ab = ba, bc = cb, ac = ca$.
                \\If $b^{2} = a$:\\
                By Claim 3, $bc = e = cb$, $c^{2} = a$. In this case, $ab = ba, bc = cb, ac = ca$.
                \\If $a^{2} = b$:\\
                By Claim 4, $ac = e =ca$. By Claim 3, $ab = c = ba$.
                \\If $b^{2} = e$:\\
                By Claim 3, $bc = a = cb$, $c^{2} = e$. In this case, $ab = ba, bc = cb, ac = ca$.
                \\If $b^{2} = a$:\\
                By Claim 3, $bc = e = cb$, $c^{2} = a$. In this case, $ab = ba, bc = cb, ac = ca$.
                \\For $a^{2} = c$, this case is analogus to case $a^{2} = b$
                \\
                We conclude that if G has at most elements, G must be abelian.
            \item
                For any $x, y \in G$, note that $xxyy = ee = e$. Also, $(xy)(xy) = e$. 
                Then, $xxyy = xyxy$ and so $xy = yx$.
        \end{enumerate}
\end{enumerate}
\end{document}
