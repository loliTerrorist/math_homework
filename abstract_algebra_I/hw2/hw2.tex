\documentclass[A4paper,12pt]{article}
\usepackage[utf8]{inputenc}
\usepackage[margin=2cm]{geometry}
\usepackage{amsmath}
\usepackage{amssymb}
\usepackage{enumerate}
\usepackage{amsthm}
\usepackage{graphicx}
\usepackage{comment}
\usepackage{xcolor}

\theoremstyle{definition}
\newtheorem{definition}{Definition}[section]
\newtheorem{theorem}{Theorem}[section]
%\theoremstyle{bfremark}
\newtheorem{remark}{Remark}[section]
\newtheorem{exmp}{Example}[section]
\newtheorem{corollary}{Corollary}[section]
\newtheorem{proposition}{Proposition}[section]
\newtheorem{lemma}{Lemma}[section]
\newtheorem{claim}{Claim}

\setlength{\parindent}{0cm}
\numberwithin{equation}{section}

\title{Abstract Algebra hw2}
\author{Chengyu Hsieh, B13201053}
\date{}
\begin{document}
\maketitle
\begin{enumerate}[1)]
    \item
        The identity element of $G$ is $0$.
        We assume 2) is proven(the proof is given below at 2)).
        \begin{enumerate}[(a)]
            \item
                $1, 14 \in G_1$, but $1+14 = 15 \not \in G_1$, so it is not close under addition $\Rightarrow$ not a subgroup.
            \item
                $G_2$ is non empty.
                For any $g_1, g_2 \in G_2$, $g_1 = 2a,\, g_2 = 2b$, we have $g_1+g_2 = 2(a+b) \in G_2$, so $G_2$ is closed under addition.
                \\
                For any $g_1 = 2a \in G_2$, $0\leq a \leq 14$, we have $g_1 + 2(15-a) = 0$. Note $2(15-a) \in G_2$, hence it is $-g \in G_2$.
                \\
                Hence by 2) $G_2$ is a subgroup of $G$.
            \item
                The identity element $0 \not \in G_3$, so it is not a subgroup of $G$.
        \end{enumerate}

    \item
        \begin{enumerate}[(i)]
            \item
                For any $a \in H$, since $a^{-1} \in H$ and $H$ is closed under $*$, we have $e = a*a^{-1} \in H$. Combined with the given criterion and that $H \subset G$, we have the required properties for a group. Hence $H \leq G$.
            \item
                Note we have $det(AB) = (detA)(detB)$. 
                The identity matrix has determinant $1$ so $S\!L_n(\mathbb{R})$ is non empty.
                For any $A,\, B \in S\!L_n(\mathbb{R})$, $det(AB) = 1 \times 1 = 1$, so $S\!L_n(\mathbb{R})$ is closed. 
                For any $A \in S\!L_n$, since $A^{-1}$ exists and $det(A^{-1}) = \frac{1}{det(A)} = 1$ so $A^{-1} \in S\!L_n$. By the criterions in (i), we know $S\!L_n(\mathbb{R}) \le G\!L_n(\mathbb{R})$.

        \end{enumerate}

    \item
        \begin{enumerate}[(a)]
            \item
                Composites of bijections are clearly bijections.
                Also the composition of functions satisfy associativity.
                The identity mapping is $I(x) = x \: \forall x \in G$.
                For any mapping $f(x) \in S_n$, since $f$ is bijective, 
                $\exists f^{-1}(x) \in S_n$, the inverse mapping of $f$, is the  inverse of $f$ in $S_n$, satisfying $f(f^{-1}(x)) = I(x)$.
                Hence $S_n$ is a group. For its order, $f(1)$ has $n$ possible values. After assigning $f(1)$, $f(2)$ has $n-1$ possible values, etc. Hence there are $n(n-1)\dots(2)(1) = n!$ elements in $S_n$.
            \item
                The closure is trivial. The inverse is just the inverse mapping. Hence $S := \{\sigma \in S_4 | \sigma(1) = 1\}$ is a subgroup of $S_4$.
            \item
                The identity of the given group is $I =
                \begin{pmatrix}
                    1 & 0\\
                    0 & 1
                \end{pmatrix}
                $.
                By calculation we find $a^4 = I$ and $b^3 = I$. Hence $a$ has order $4$ and $b$ has order $3$.
        \end{enumerate}
    \item
        \begin{claim}
            For any $x \in G$, $g \in G$ a generator of G, we have $x = xg^{bn}$ for any $b \in \mathbb{Z}$.
            \proof
            $g^n = e$, so $g^bn = (g^n)^b = e^b = e$ for any $b \in \mathbb{Z}$.
        \end{claim}

        \begin{claim}
            Let $a, b \in \mathbb{Z},\, a,b>0$. If $\exists c,d\in \mathbb{Z} \text{ such that } ca+db=1,\, \text{then} (a,b)=1.$
            \proof
            Suppose $(a, b) = k>1$. Let $a = k\alpha,\, b = k\beta$. Let $c, d$ be integers such that $ca+db=1$. Subsitute in $k\alpha \text{ and } k\beta$ we obtain
    `        
            \begin{align*}
                &k(c\alpha+d\beta) = 1
            \end{align*}
            Since $k \text{ and }c\alpha+d\beta$ are integers, we have $(k=1 \land c\alpha+d\beta = 1) \lor (k = -1 \land c\alpha+d\beta = -1)$. But we assumed that $k>1$ and so we arrived at an contradiction.
            Hence $(a,b) = 1$.
        \end{claim}
        \begin{claim}
            Let $a, b \in \mathbb{Z},\, a,b>0$. If $(a,b)=m$, then exists $c, d \in \mathbb{Z} \text{ such that } ca+db = m$.
            \proof
            First note that if $d|a \land d|b$, then $d|m$. Now let $M = \{ca+db|c, d \in \mathbb{Z}\}$. $1a+0b = a > 0$, so $M$ has positive integers. Let $M^{+} = \{ca+db>0|c, d \in \mathbb{Z}\}$, which is non empty. Clearly $\min{M^{+}}$ exist. Let $m' = \min{M^{+}}$. Then $m' = c'a+d'b$ for some $c', d' \in \mathbb{Z}$. Now for any $x = ca+db \in M$, let $x = m'q+r$ with $0 \leq r <m'$.
        \begin{align*}
            &r = x-m'q = (c-c'q)a + (d-d'b)b \in M
        \end{align*}
        Since we have $0 \leq r < m'$, we have $r=0$. (Otherwise $m' \neq \min(M^{+}))$.)
        Therefore $m'|x \; \forall x \in M$. Note that $a, b\in M \Rightarrow m'|a \land m'|b$. Also, for any $d \text{ such that } d|a \land d|b$ we have $d|c'a+d'b$, so $d|m'$. Hence $m' = (a,b) = m$.
        We conclude that $\exists c, d \in \mathbb{Z} \text{ such that } ca+db = m$.
        \end{claim}
        \begin{enumerate}[(a)]
            \item
                If $n=1$ then $g^k = e$ is always a generator, also $gcd(k, n) = 1$ for any k.
                We then consider $n\ge 2$\\
                \textbf{Sufficiency:}\\
                Suppose $g^k$ is a generator of $G$. Since $g^k$ is a generator, $g = (g^k)a$ for some $a$. 
                Let $ka = bn + m$, $1 \le m \le n-1$($m$ cannot be zero as that leeds to $g = e$ which cannot be true for $n \ge 2$.)
                Then by Claim 1, $g = (g^k)^a = g^{ak-bn} = g^m$.
                Applying $g^{-1}$ to both sides, we obtain $g^{m-1} = e$.
                If $m-1 \neq 0$ then we found $n' = m-1 < n$ such that $g^{n'} = e$, contradicting the fact that $g$ is a generator of G.
                Hence $m = 1$. Then by Claim 2, $\exists a,\, b \in \mathbb{Z}$ such that $ak + (-b)n = 1$, so $gcd(n, k) = 1$.
                \\
                \textbf{Necessity: }\\
                Suppose $gcd(k,n) = 1$. Then by Claim 3, $\exists a,\, b \in \mathbb{Z}$ such that $ak + bn = 1$. 
                So $(g^k)^a = g^{ak} = g^{1-bn} = g^1 = g$. Clearly this leads to $g^k$ also being a generator.
            \item
                Let $n = ma$. Consider $S = \{g^a,\, g^{2a},\, \dots, g^{ma}\}$.
                Note that $0<a<2a<\dots<(m-1)a<ma=n$, so $|S| = m$. For any $g^{\gamma a} \in S$, see that $g^{\gamma a} g^{(m-\gamma)a} = e$ and $g^{(m-\gamma)a}$ is clearly in $S$. Also $S$ is clearly closed. 
                Thus $S \le G$ and we have found a subgroup of order $m$.
                \\
                Now suppose we find $H\le G$ with $|H| = m$.
                Let $c$ be the smallest positive integer such that $g^c \in H$. 
                If $\exists g^x  \in H$ with $x = yc + z$, $1\le z \le c-1$, we have $g^z = g^xG^{-yc} \in H$. But $z<c$ so we have arrived at a contradiction.
                Hence $\forall g^x \in H$, we have $g^x = (g^c)^y$ for some y, and so $g^c$ generates $H$. 
                Now, $g^{mc} = g^n$ so $mc = bn = bam$ for some $b$.
                Hence $c = ab$. This gives us $g^c = (g^a)^b \in H$. 
                But $c$ is the smallest positive integer such that $g^c \in H$.
                So $a = c$ and $H = S$.
                \\
                We conclude that G has exactly one subgroup of order $m$.
            \item
                $S_3 = \{(1), (12), (13), (23), (123), (132)\}$ has order 6. 
                Order of $(1)$ is $1<6$, order of $(12), (13), (23)$ are all $2<6$, order of $(123), (132)$ are both $3<6$. Hence no elements of $S_3$ g can generate the entire group, so $S_3$ is not cyclic.
        \end{enumerate}



    \item
        \begin{enumerate}[(a)]
            \item
                For any $g_1N, g_2N \in G/N$, since $g_1 * g_2 \in G$, we have $g_1N \cdot g_2N \in G/N$. 
                For $g_1N, g_2N, g_3N \in G/N$, $(g_1N \cdot g_2N) \cdot g_3N = (g_1 * g_2)N \cdot g_3N = (g_1 * g_2*g_3) N = g_1N \cdot (g_2*g_3)N = g_1N \cdot (g_2N \cdot g_3N)$, thus associativity holds.
                $N = eN \in G/N$ is clearly the identity.
                For any $gN \in G/N$, $g^{-1}N$ is clearly in $G/N$ and is the inverse of $gN$.
            \item
                $(13)H = \{(13), (123)\}$, $H(13) = \{(13), (132)\} \neq (13)H$. Thus $H$ is not a normal subgroup of $S_3$.
            \item
                Let $H \le G$. $\forall h \in H$, $g \in G$, we have $g * h = h*g$ so $gH = \{g*h| h\in H ,\: g \in G\} = \{h*g| h\in H ,\: g \in G\} = Hg$. 
                So any subgroup of an abelian group G is normal.
            \item
                Consider the quaterion group $Q_4 = {\pm1, \pm i, \pm j, \pm k}$ with $i^2 = j^2 = k^2 = -1$ and $ij = -ji = k$, $jk = -kj = i$, $ki = -ik = j$.
                This group satisfies the required conditions. 
        \end{enumerate}
\end{enumerate}

\end{document}
