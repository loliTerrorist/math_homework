\documentclass[A4paper,12pt]{article}
\usepackage[utf8]{inputenc}
\usepackage[margin=2cm]{geometry}
\usepackage{amsmath}
\usepackage{amssymb}
\usepackage{enumerate}
\usepackage{amsthm}
\usepackage{graphicx}
\usepackage{comment}
\usepackage{xcolor}

\theoremstyle{definition}
\newtheorem{definition}{Definition}[section]
\newtheorem{theorem}{Theorem}[section]
%\theoremstyle{bfremark}
\newtheorem{remark}{Remark}[section]
\newtheorem{exmp}{Example}[section]
\newtheorem{corollary}{Corollary}[section]
\newtheorem{proposition}{Proposition}[section]
\newtheorem{lemma}{Lemma}[section]
\newtheorem{claim}{Claim}

\setlength{\parindent}{0cm}
\numberwithin{equation}{section}

\title{Abstract Algebra hw4}
\author{Chengyu Hsieh, B13201053}
\date{}
\begin{document}
\maketitle
\begin{enumerate}[1)]
    \item
        \begin{enumerate}[(a)]
            \item
                We assume that (b) is proven and use its proof and result. 
                We have
                $$|\mathbb{Z}/p\mathbb{Z}| = p \quad \text{and} \quad |\mathbb{Z}/q\mathbb{Z}| = q$$
                Note that both $\mathbb{Z}/p\mathbb{Z}$ and $\mathbb{Z}/q\mathbb{Z}$ are cyclic.

                Since $(p, q) = 1$, the direct product of the groups, $\mathbb{Z}/p\mathbb{Z} \times \mathbb{Z}/q\mathbb{Z}$, is also cyclic, with $(1, 1)$ as its generator.
                Its order is the product of the individual orders:
                $$|\mathbb{Z}/p\mathbb{Z} \times \mathbb{Z}/q\mathbb{Z}| = pq$$
                The order of $\mathbb{Z}/pq\mathbb{Z}$ is also $pq$:

                Since both $\mathbb{Z}/p\mathbb{Z} \times \mathbb{Z}/q\mathbb{Z}$ and $\mathbb{Z}/pq\mathbb{Z}$ are cyclic groups of the same order $pq$, they are isomorphic.
                Now, the map $\phi$ that sends the generator of the direct product to the generator of $\mathbb{Z}/pq\mathbb{Z}$:
$$\phi: (1, 1) \mapsto 1$$ is clearly an isomorphism.
            \item
                Assume that $G \times H$ is cyclic. Let $(g, h)$ generate $G \times H$. 
                Let $(|G|, |H|) = k$, $|G| = ck$, $|H|=dk$. Then $[|G|, |H|] = cdk$.
                Clearly, $(g, h)^{cdk} = (e, e)$. Since $(g, h)$ is the generator, we must have $cdk \ge |G \times H| = |G||H| = cdk^2$.
                But also $cdk \le cdk^2$ so $cdk = cdk^2$ and hence $k=1$.
                \\
                Now assume that $(|G|,|H|)= 1$. 
                Let $g$ generate $G$ and $h$ generate $H$.
                Then for any $i$ such that $(g, h)^i= (e, e)$, we must have $|G| \mid i$ and $|H| \mid i$.
                From this we have $|G||H| = [|G|, |H|] \mid i$ and $i \ge |G||H| = |G \times H|$. 
                Also note that $(g, h)^{|G||H|} = (e,e)$. Hence $(g, h)$ fullfill the properties of a generator for the group, and so $G \times H$ is cyclic.
            \item
                The proper subgroups of $S_3$ are $\{e, r, r^2\}$, $\{e, s\}$, $\{e, rs\}$, $\{e, r^2s\}$  and ${e}$. All of those proper subgroups are cyclic, and by (b) if $S_3$ is a direct product of some of its proper subgroup then it should be cyclic, but $S_3$ is not cyclic  and hence it cannot be a direct product of any of its proper subgroups.
        \end{enumerate}
    \item
        $G = \{e, a, a^2, a^3, a^4, a^5, b, ba, ba^2, ba^3, ba^4, ba^5\}$, which has order $12$.  
        See that $ba$ has order $4$. However the elements in $H$ can only be of order $1, 2, 3, 6$. Hence the two groups are not isomorphic.
    \item
        \begin{enumerate}[(a)]
            \item
                Orbit of $\langle (12) \rangle$ is $\{\{1, 2\}, \{3\}, \{4\}\}$.\\
                Orbit of $\langle (123) \rangle$ is $\{\{1, 2, 3\}, \{4\}\}$.\\
                Orbit of $V$ is $\{\{1, 2, 3, 4\}\}$.
            \item
                $\langle (1234) \rangle$ has the same orbit.
            \item
                Suppose that $s \in Z(S_4)$. 
                $((12)s) (3) = ((12)s(12)) (3) = s(3)$. 
                Hence $s(3) \neq 1, 2$. Similarily, using $(14), (24)$ we obtain that $s(3) \neq 1, 2, 4$. Hence $s(3) = 3$. 
                Similarily we obtain $s(i) = i$ for $i = 1, 2, 3, 4$, therefore $s$ is the trivial permutation. We conclude that $Z(S_4)$ is trivial.

        \end{enumerate}
    \item
        \begin{enumerate}[(a)]
            \item
                For any $g \in G$, $s_1, s_2 \in S$, if $g(s_1) = g(s_2)$ then $s_1 = g^{-1}g(s_1) = g^{-1}g(s_2) = s_2$, hence $g(\cdot)$ is injective.
                Also for any $s \in S$, $g(g^{-1}(s)) = s$, hence $g(\cdot)$ is surjective. We conclude that $g(\cdot)$ is bijective and may generate a permutation of $S$.
                Let $\phi: G \rightarrow \mathrm{Perm}(S)$ be defined as $\phi(g) = $ the permutation generated by $g(\cdot)$.
                Now for any $s \in S$, $g, h \in G$, \begin{align*}
                    &g(h(s)) = (gh)(s) \\
                    \Rightarrow &\phi(g)\phi(h) = \phi(gh)
                \end{align*}
                Hence $\phi$ is a homomorphism.
            \item
                If $\phi(g') = \phi(e)$, then $\phi(g')(H) = H$. Then we have $g'H = H$ and hence $g' \in H$. We conclude that the kernel of $\phi$ is in $H$.
            \item
                Note that $|S| = |G| / |H| = n$. Hence $S_n \simeq \mathrm{Perm}(S)$.
                From (b) we know that $\ker \phi \subseteq H$. But $\ker \phi$ is a normal subgroup of $G$, and adding the condition that (c) gives we see that $\ker \phi$ must be trivial, meaning that $\phi$ is injective. We now define $\phi': G \rightarrow \mathrm{Im}(\phi)$ by $\phi'(g) = \phi(g)$. 
                Then $\phi'$ is clearly an isomorphism. Hence $G$ is isomorphic to a subgroup of $S_n$.
        \end{enumerate}


\end{enumerate}

\end{document}
