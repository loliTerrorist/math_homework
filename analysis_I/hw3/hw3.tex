\documentclass[A4paper,12pt]{article}
\usepackage[utf8]{inputenc}
\usepackage[margin=2cm]{geometry}
\usepackage{amsmath}
\usepackage{amssymb}
\usepackage{enumerate}
\usepackage{amsthm}
\usepackage{graphicx}
\usepackage{comment}
\usepackage{xcolor}

\theoremstyle{definition}
\newtheorem{definition}{Definition}[section]
\newtheorem{theorem}{Theorem}[section]
%\theoremstyle{bfremark}
\newtheorem{remark}{Remark}[section]
\newtheorem{exmp}{Example}[section]
\newtheorem{corollary}{Corollary}[section]
\newtheorem{proposition}{Proposition}[section]
\newtheorem{lemma}{Lemma}[section]
\newtheorem{claim}{Claim}

\setlength{\parindent}{0cm}
\numberwithin{equation}{section}

\title{Analysis HW3}
\author{}
\date{}
\begin{document}
\maketitle
\begin{enumerate}[(1)]
    \item
        Let $L$ be a limit point of the sequence $(x^{(n)})_{n=m}^\infty$.
        Consider $B(L, \epsilon)$ for any $\epsilon>0$. 
        Then $\exists  n \ge m$ such that $d(x^{(n)}, L) < \epsilon \Rightarrow x^{(n)} \in B(L, \epsilon) \cap S$. 
        So $B(L, \epsilon)  \cap S \neq \varnothing$ and $L$ is an adherent point of $S$.
        \\
        Now consider the sequence $x_1 = 1, x_i = 0$ for $i\ge 2$.
        Then $\overline{S} = \{1, 0\}$
        Set $L = 1$.
        For $\epsilon = \frac{1}{2} > 0$, $N = 2$,  see that $\forall n\ge N$ we have 
        \begin{align*}
            d(x^{(n)}, L) = d(0, 1) = 1 \not < \frac{1}{2}
        \end{align*}
        . Hence $L$ is not a limit point of this sequence and the converse is false.
    \item
        \begin{enumerate}[(a)]
            \item
                \textbf{Reflexive: }\\
                For some Cauchy sequence $(x_n)_{n=1}^\infty$ in X, 
                $\lim_{n \rightarrow \infty} d(x_n, x_n) = \lim_{n \rightarrow \infty} 0 = 0$.
                Hence $\mathrm{LIM}_{n \rightarrow \infty} x_n = \mathrm{LIM}_{n \rightarrow \infty} x_n$.
                \\
                \textbf{Symmetry: }\\
                For some Cauchy sequences $(x_n)_{n=1}^\infty$, $(y_n)_{n=1}^\infty$ in X such that $\mathrm{LIM}_{n\rightarrow \infty} x_n = \mathrm{LIM}_{n\rightarrow \infty} y_n$, 
                $\lim_{n \rightarrow \infty} d(x_n, y_n) = \lim_{n \rightarrow \infty} d(y_n, x_n)$.  
                Hence $\mathrm{LIM}_{n \rightarrow \infty} y_n = \mathrm{LIM}_{n \rightarrow \infty} x_n$.
                \\
                \textbf{Transitive: }\\
                For some Cauchy sequences $(x_n)_{n=1}^\infty$, $(y_n)_{n=1}^\infty$, $(z_n)_{n=1}^\infty$ in X such that $\mathrm{LIM}_{n \rightarrow \infty} x_n = \mathrm{LIM}_{n \rightarrow \infty} y_n$ and $\mathrm{LIM}_{n \rightarrow \infty} y_n = \mathrm{LIM}_{n \rightarrow \infty} z_n$, 
                note that $0 \le \lim_{n \rightarrow \infty} d(x_n,z_n) \le \lim_{n \rightarrow \infty} (d(x_n, y_n)+ d(y_n, z_n)) = 0$ by the triangle inequality.
                Hence $\lim_{x  \rightarrow \infty } d(x_n, z_n)= 0$ and $\mathrm{LIM}_{x \rightarrow \infty} x_n  = \mathrm{LIM}_{n \rightarrow \infty} z_n$.
            \\We conclude that the equality relation of the formal limit is an equivalence relation.
            \item
                We first verify that the limit exists. 
                Given two Cauchy sequences $(x_n)_{n=1}^\infty$, $(y_n)_{n=1}^\infty$ in $X$, by definition we can find $m>n$ such that $d(x_n, x_m)<\frac{\epsilon}{2}$ and $d(y_n, y_m) < \frac{\epsilon}{2}$. 
                Then, by the triangle inequality we obtain $d(x_m,y_m) - d(x_n, y_n) \le d(x_n, x_m) + d(y_n, y_m) < \epsilon$. Thus $(d(x_n, y_n))_{n=1}^\infty$ is Cauchy in $\mathbb{R}$ with the usual metric, which is complete, so $\lim_{n\rightarrow \infty} d(x_n, y_n)$ exists.
                \\
                Next we show that this distance does not depend on the choice of representatives.
                Let $(x_n)_{n=1}^\infty$, $(x_n')_{n=1}^\infty$, $(y_n)_{n=1}^\infty$ be Cauchy sequences in $X$ such that
                $\mathrm{LIM}_{n \rightarrow \infty} x_n = \mathrm{LIM}_{n \rightarrow \infty} x_n'$. 
                Then $\lim_{n \rightarrow \infty} d(x'_n, y_n) \le  \lim_{n \rightarrow \infty}(d(x_n', x)+d(x_n, y_n)) = \lim_{n \rightarrow \infty}d(x_n, y_n)$.
                But also, $\lim_{n \rightarrow \infty} d(x_n, y_n) \le \lim_{n \rightarrow \infty}(d(x_n, x_n')+d(x_n', y_n)) = \lim_{n \rightarrow \infty} d(x_n', y_n)$. 
                Hence we have $\lim_{n \rightarrow \infty} d(x_n, y_n) = \lim_{n \rightarrow \infty} d(x'_n, y_n) \\ \Rightarrow d_{\overline{X}}(\mathrm{LIM}_{n \rightarrow \infty}x_n, \mathrm{LIM}_{n \rightarrow \infty}y_n) = d_{\overline{X}}(\mathrm{LIM}_{n \rightarrow \infty}x_n', \mathrm{LIM}_{n \rightarrow \infty}y_n)$.
                \\
                Now we show that $d_{\overline{X}}$ is a metric.\\
                For any Cauchy sequence $(x_n)_{n = 1}^\infty$ in $X$, $d_{\overline{X}}(\mathrm{LIM}_{n \rightarrow \infty} x_n, \mathrm{LIM}_{n \rightarrow \infty} x_n) = \lim_{n \rightarrow \infty} d(x_n, x_n) = 0$.
                \\
                For any Cauchy sequences $(x_n)_{n = 1}^\infty$, $(y_n)_{n = 1}^\infty$ in $X$ such that $\mathrm{LIM}_{n \rightarrow \infty}(x_n) \neq \mathrm{LIM}_{n \rightarrow \infty}(y_n)$,
                $d_{\overline{X}}(\mathrm{LIM}_{n \rightarrow \infty}x_n,\mathrm{LIM}_{n \rightarrow \infty}y_n) = \lim_{n \rightarrow \infty} d(x_n, y_n) > 0$.
                \\
                For any Cauchy sequences $(x_n)_{n = 1}^\infty$, $(y_n)_{n = 1}^\infty$ in $X$, $d_{\overline{X}}(\mathrm{LIM}_{n \rightarrow \infty}x_n,\mathrm{LIM}_{n \rightarrow \infty}y_n ) = \lim_{n \rightarrow \infty} d(x_n, y_n)\\ = \lim_{n \rightarrow \infty} d(y_n, x_n) = d_{\overline{X}}(\mathrm{LIM}_{n \rightarrow \infty}y_n,\mathrm{LIM}_{n \rightarrow \infty}x_n )$
            \item
                
            \item

            \item

            \item
        \end{enumerate}
    \item
        \begin{enumerate}[(a)]
            \item
                $\partial(A \cup B) \supseteq \partial A \cup \partial B$ is trivial. 
                We prove that $\partial(A \cup B) \subseteq \partial A \cup \partial B$.
                Suppose $\exists x \in \partial(A\cup B)$ such that $x \not \in \partial A \cup \partial B$.
                $\forall r>0$, $B(x, r) \cap (A\cup B) \neq \varnothing$
                .
                Also, $\overline{A} \cap \overline{B} =  \varnothing$, so for a given $r$, exactly one of $B(x, r) \cap A \neq \varnothing$, $B(x, r) \cap B \neq \varnothing$ must be true.
                Since $x \not \in \partial A$, $\exists r_1 > 0$ such that $B(x, r_1)\cap A = \varnothing$. Then $B(x, r_1) \cap B \neq \varnothing$.
                However, $B(x, r_1) \subseteq B(x, r)$ for any $r \ge r_1$. 
                Thus $B(x, r) \cap B \neq \varnothing$ and $B(x, r) \cap A = \varnothing$ for any $r \ge r_1$.
                Thus $\partial(A  \cup B) \subseteq \partial A \cup \partial B$ and so $\partial(A  \cup B) = \partial A \cup \partial B$.
                Now note that $x \not \in \partial B$, so $\exists 0<r_2<r_1$ such that $B(x, r) \cap B = \varnothing$. But then $B(x, r_2) \cap A \neq \varnothing$ and $B(x, r_1)\cap A \supseteq B(x, r_2) \cap A \neq \varnothing$, arriving at a contradiction.
                Hence $\partial(A\cup B) \subseteq \partial A \cup \partial B$ and $\partial(A\cup B) = \partial A \cup \partial B$.
            \item

        \end{enumerate}
    \item
    \item
    \item


\end{enumerate}

\end{document}
