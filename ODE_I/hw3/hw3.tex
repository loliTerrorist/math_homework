\documentclass[A4paper,12pt]{article}
\usepackage[utf8]{inputenc}
\usepackage[margin=2cm]{geometry}
\usepackage{amsmath}
\usepackage{amssymb}
\usepackage{enumerate}
\usepackage{amsthm}
\usepackage{graphicx}
\usepackage{comment}
\usepackage{xcolor}

\theoremstyle{definition}
\newtheorem{definition}{Definition}[section]
\newtheorem{theorem}{Theorem}[section]
%\theoremstyle{bfremark}
\newtheorem{remark}{Remark}[section]
\newtheorem{exmp}{Example}[section]
\newtheorem{corollary}{Corollary}[section]
\newtheorem{proposition}{Proposition}[section]
\newtheorem{lemma}{Lemma}[section]
\newtheorem{claim}{Claim}

\setlength{\parindent}{0cm}
\numberwithin{equation}{section}

\title{ODE HW3}
\author{Chengyu Hsieh, B13201053}
\date{}
\begin{document}
\maketitle

\textbf{I.} Give the solutions, where possible in terms of the Bessel
functions, of the differential equations
\begin{enumerate}
\item [(a)]\(x\frac{d^2y}{dx^2}+(x+1)^2y=0\),
\item [(b)]\((1-x^2)\frac{d^2y}{dx^2}-2x\frac{dy}{dx}
+n(n+1)y=0\)
\end{enumerate}
\textbf{Grading:}
[1 \%]: State your method and explain why it works.
[2 \%]: Show your calculations.
\\
\textbf{Sketch: }
For (a), use the method of Frobenius, we would find that the roots of the indicial equation differ by 1.
For (b), the equation is clearly a legendre equation. Use the formula for the legendre functions, then use the reduction of order to obtain the other solution.\\
\textbf{Calculation:}
\begin{enumerate}[(a)]
    \item
        Start by assuming $y = \sum_{r=0}^\infty a_r x^{r+c}$. 
        Plug it into the equation and obtain 
        \begin{align*}
            &\sum_{r=0}^\infty a_r(r+c)(r+c-1)x^{r+c-1} + (x^2 + 2x + 1)\sum_{r=0}^\infty a_r x^{r+c} = 0\\
            &\text{For easier notation let }a_{-1} = a_{-2} = 0.\\
            &a_0 (c)(c-1)x^{c-1} + \sum_{r=0}^\infty a_{r+1}(r+c+1)(r+c)x^{r+c} +   \sum_{r=0}^\infty a_rx^{r+c} + \sum_{r=0}^\infty 2a_{r-1} x^{r+c}+ \sum_{r=0}^\infty a_{r-2}x^{r+c} = 0\\
            &a_0c(c-1) = 0\\
            &\text{Note that if $c=0$, this makes $(c)(c+1)a_1+a_0 = 0$ for any $a_1$. This gives $a_0 = 0$ which we do not want.} \\
            &\text{Thus we let $c=1$.}
        \end{align*}
    \item
        $P_n(x) = \sum_{r=0}^m (-1)^r \frac{(2n-2r)!x^{n-2r}}{2^nr!(n-r)!(n-2r)!}$
\end{enumerate}
\vspace{20pt}
\textbf{II.} Determine the coefficients of the Fourier-Bessel series
for the function
\begin{align*}
f(x)=
\begin{cases}
1 \quad \text{ for } 0 \leq x< 1,\\
-1 \quad \text{ for } 1 \leq x \leq 2,
\end{cases}
\end{align*}
in terms of the Bessel function \(J_0(x)\).\\
\textbf{Grading:}
[1 \%]: State your method and explain why it works.
[2 \%]: Show your calculations.

\textbf{Sketch:}\\
By \cite{textbook}, the coefficient $C_j$ corresponding to $J_0(\lambda_j x)$ is $\frac{2}{a^2[J_0'(\lambda_ja)]^2} \int_0^a xJ_0(\lambda_jx)f(x)dx$.\\
\textbf{Calculations:}\\
\begin{align*}
    C_j &= \frac{2}{2^2[J_0'(\lambda_ja)]^2} \int_0^a xJ_0(\lambda_jx)f(x)dx
    \\
        &= \frac{1}{2[J_0'(\lambda_ja)]^2}(\int_0^1 xJ_0(\lambda_jx)dx - \int_1^2 xJ_0(\lambda_jx)dx)
        \\
        &\text{Let } y = \lambda_jx. \\
    C_j &= \frac{1}{2\lambda_j^2[J_0'(\lambda_ja)]^2}(\int_0^1 xJ_0(\lambda_jx)dx - \int_1^2 xJ_0(\lambda_jx)dx)
        \\
        &= \frac{1}{2\lambda_j^2[J_0'(\lambda_ja)]^2}(\int_0^{\lambda_j} yJ_0(y)dy - \int_{\lambda_j}^{2\lambda_j} yJ_0(y)dy)
        \\
        &\text{ By }\cite{textbook},\; \frac{x}{dx}(x^cJ_c(x)) = x^cJ_{c-1}(x)\\
    C_j &= \frac{1}{2\lambda_j^2[J_0'(\lambda_ja)]^2}(\int_0^{\lambda_j} \frac{d}{dy}(yJ_1(y))dy - \int_{\lambda_j}^{2\lambda_j} \frac{d}{dy}(yJ_1(y))dy)
        \\
        &= \frac{1}{2\lambda_j^2[J_0'(\lambda_ja)]^2}(\lambda_jJ_1(\lambda_j)-(2\lambda_j J_1(2\lambda_j)-\lambda_jJ_1(\lambda_j))
        \\
        &= \frac{1}{2\lambda_j^2[J_0'(\lambda_ja)]^2}(2\lambda_jJ_1(\lambda_j)-2\lambda_j J_1(2\lambda_j))
        \\
        &\text{By }\cite{textbook},\; J'_0(x) =  -J_1(x)\\
    C_j &= \frac{1}{\lambda_j[J_1(2\lambda_j)]^2}(J_1(\lambda_j)-J_1(2\lambda_j))
\end{align*}

\bibliographystyle{plain}
\bibliography{refs}
\end{document}
