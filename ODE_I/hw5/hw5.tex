\documentclass[A4paper,12pt]{article}
\usepackage[utf8]{inputenc}
\usepackage[margin=2cm]{geometry}
\usepackage{amsmath}
\usepackage{amssymb}
\usepackage{enumerate}
\usepackage{amsthm}
\usepackage{graphicx}
\usepackage{comment}
\usepackage{xcolor}

\theoremstyle{definition}
\newtheorem{definition}{Definition}[section]
\newtheorem{theorem}{Theorem}[section]
%\theoremstyle{bfremark}
\newtheorem{remark}{Remark}[section]
\newtheorem{exmp}{Example}[section]
\newtheorem{corollary}{Corollary}[section]
\newtheorem{proposition}{Proposition}[section]
\newtheorem{lemma}{Lemma}[section]
\newtheorem{claim}{Claim}

\setlength{\parindent}{0cm}
\numberwithin{equation}{section}


\title{ODE Homework 5\, (6 \%)}
\author{Chengyu Hsieh, B13201053}
\date{}
\begin{document}
\maketitle
\textbf{I.}[2 \%]
Using Laplace transforms, solve the initial value problem
\begin{align*}
\frac{dx}{dt}+x+\frac{dy}{dt}=0,\quad \frac{dx}{dt}-x+2\frac{dy}{dt}=e^{-t}, \quad \text{subject to } x(0)=y(0)=1.
\end{align*}
\textbf{Grading:}
[1 \%]: State your method and explain why it works.
[1 \%]: Show your calculations.
\vspace{20pt}
\\
\textbf{Method:}\\
The Laplace transform is linear, hence we can apply it to each terms in the equation. 
The resulting equation is easy to solve, and we apply the inverse Laplace transform to the solution.
\\
\textbf{Calculations:}\\
Let $\mathcal{L}[x] = X(s)$ and $\mathcal{L}[y] = Y(s)$.
Applying the Laplace transformation gives:
\begin{align*}
sX(s) - x(0) + X(s) + sY(s) - y(0) &= 0 \\
sX(s) - x(0) + X(s) + 2sY(s) - 2y(0) &= \frac{1}{s+k}
\end{align*}
Substitute in $x(0) = y(0) = 0$ we obtain:
\begin{align*}
(s+1)X(s) + sY(s) &= 2 \\
(s-1)X(s) + 2sY(s) &= 3+\frac{1}{s+1}
\end{align*}
Solving the equation gives:
\begin{align*}
X(s) &= \frac{1}{s+3} - \frac{1}{(s+1)(s+3)} \\
&= \frac{3}{2} \frac{1}{s+3} - \frac{1}{2} \frac{1}{s+1}
\end{align*}
Apply the inverse Laplace transform gives:
$$x = \mathcal{L}^{-1}[X(s)] = \frac{3}{2} e^{-3t} - \frac{1}{2} e^{-t}$$
For $Y(s)$:
$$Y(s) = \frac{2}{s} - \frac{1}{s+3}$$
Apply the inverse Laplace transform gives:
$$y = \mathcal{L}^{-1}[Y(s)] = 2-e^{-3t}$$
\\

\textbf{II.}[2 \%] Show that \(\mathcal{L}[tf(t)]=-\frac{dF}{ds}\),
where \(F(s)=\mathcal{L}[f(t)]\). Hence solve the initial value
problem
\begin{align*}
\frac{d^2x}{dt^2}+2t\frac{dx}{dt}-4x=1, \quad \text{subject
to }x(0)=x'(0)=0.
\end{align*}
\textbf{Grading:}
[1 \%]: State your method and explain why it works.
[1 \%]: Show your calculations.
\vspace{20pt}
\\
\textbf{Methods: }\\
Notice that 
\begin{align*}
    \mathcal{L}[tf(t)] &= \int_0^\infty tf(t) e^{-st}dt \\
                       &= -\int_0^\infty \frac{d}{ds}(e^{-st}f(t))dt \\
                       &= -\frac{d}{ds} \int_0^\infty e^{-st}f(t)dt \\
                       &= -\frac{dF}{ds}
\end{align*}
By \cite{textbook}, we have 
$$\mathcal{L}[f'(t)] = s\mathcal{L}[f(t)] - f(0)$$
$$\mathcal{L}[f''(t)] = s^2\mathcal{L}[f(t)] - sf(0) - f'(0)$$
Substitute the above into the equation and solve for $F(s)$, then apply the Inverse Laplace transform.
\\
\textbf{Calculations: }\\
\begin{align*}
& s^2 F(s) - s x(0) - s x'(0) - 2 ( s F(s))' - 4 F(s) = \frac{1}{s} \\
& s^2 F(s) - 2 F(s) - 2 s F'(s) - 4 F(s) = \frac{1}{s} \\
& -2 s F'(s) + (s^2 - 6)F(s) = \frac{1}{s} \\
& F'(s) - ( \frac{s}{2} - \frac{3}{s} ) F(s) = -\frac{1}{2 s^2} \\
& F'(s) + ( \frac{s^3}{2} - \frac{3}{s} ) F(s) = -\frac{1}{2s^2}
\end{align*}
This is a first order differential equation. By \cite{textcal},
Let 
\begin{align*}
    A(x) &= \int \frac{s^3}{2} - \frac{3}{s}  ds \\
         &= 3 \ln(x) - \frac{1}{4} x^4 + C_1\\
    e^{A(x)} &= C_2 x^3 e^{-x^2/4}
\end{align*}
Then we have 
\begin{align*}
F(s) &= \frac{e^{s^2/4}}{C_2 s^3} \int x^3 e^{-x^2/4} ( -\frac{1}{2x^2}) dx\\
     &= \frac{e^{s^2/4}}{s^3} \int -\frac{x}{2} e^{-x^2/4} dx \\
\text{let }u &= \frac{1}{4} x^2 \\
du &= \frac{1}{2} x dx \\
F(s) &= \frac{e^{s^4/4}}{s^3} \int e^{u} du \\
&= \frac{e^{s^4/4}}{s^3} ( e^{u} + C ) \\
&= \frac{1}{s^3} + C \frac{e^{s^2/4}}{s^3}
\end{align*}
By \cite{textbook}, in order for the transform to be valid, we must have $F(s) \rightarrow 0$ as $s \rightarrow \infty$. Hence $C= 0$.
Then $F(s) = \frac{1}{s^3}$ and
$$
x(t) = \mathcal{L}^{-1}[\frac{1}{s^3}] = \frac{1}{2!} \mathcal{L}^{-1}[\frac{2!}{s^{2+1}}] = \frac{t^2}{2}
$$.
\\
\textbf{III.}[2 \%]
Find the inverse Laplace transform of
\begin{align*}
F(s)=\frac{1}{(s-1)(s^2+1)},
\end{align*}
by (a) expressing \(F(s)\) as partial fractions and inverting the
constituent parts, and (b) using the convolution theorem.\\
\textbf{Grading:}
[1 \%]: State your method and explain why it works.
[1 \%]: Show your calculations.\\
\textbf{Methods: }\\
We follow \cite{textbook}.
\\
\textbf{Calculations: }\\
\begin{enumerate}[(a)]
    \item
        Notice that $$\frac{1}{(s-1)(s^2+1)} = \frac{1}{2(s-1)} - \frac{1}{2}(\frac{s+1}{s^2+1})$$.
        Hence $\mathcal{L}^{-1}[F] = \frac{1}{2}(e^{-t}-\cos t - \sin t)$.
    \item
        By the convolution theorem,
        \begin{align*}
            \mathcal{L}^{-1}[F] &= \mathcal{L}^{-1}[\frac{1}{s-1}] * \mathcal{L}^{-1} [\frac{1}{s^2+1}] \\
                                &= e^t * \sin t \\
                                &= \int_0^t e^\tau \sin (t-\tau) d\tau \\
                                &= -\sin t + \int_0^t e^\tau \cos(t-\tau)d\tau \\
                                &= -\sin t + e^t - \cos t - \int_0^t e^\tau \sin (t-\tau) d\tau 
        \end{align*}
        Hence $\mathcal{L}^{-1}[F] = \frac{1}{2}(e^{-t}-\cos t - \sin t)$.
\end{enumerate}

\bibliographystyle{plain}
\bibliography{refs}
\end{document}
