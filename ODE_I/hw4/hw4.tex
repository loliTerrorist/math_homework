\documentclass[A4paper,12pt]{article}
\usepackage[utf8]{inputenc}
\usepackage[margin=2cm]{geometry}
\usepackage{amsmath}
\usepackage{amssymb}
\usepackage{enumerate}
\usepackage{amsthm}
\usepackage{graphicx}
\usepackage{comment}
\usepackage{xcolor}

\theoremstyle{definition}
\newtheorem{definition}{Definition}[section]
\newtheorem{theorem}{Theorem}[section]
%\theoremstyle{bfremark}
\newtheorem{remark}{Remark}[section]
\newtheorem{exmp}{Example}[section]
\newtheorem{corollary}{Corollary}[section]
\newtheorem{proposition}{Proposition}[section]
\newtheorem{lemma}{Lemma}[section]
\newtheorem{claim}{Claim}

\setlength{\parindent}{0cm}
\numberwithin{equation}{section}

\title{ODE Homework 4}
\author{Chengyu Hsieh, B13201053}
\date{}
\begin{document}
\maketitle
\textbf{I.}[2 \%] Comment on the difficulties that you face when
trying to construct the Green's function for the boundary value
problem
\begin{align*}
y''(x)+y(x)=f(x) \quad \text{subject to} \quad y(a)=y'(b)=0.
\end{align*}
\textbf{Grading:}
[1 \%]: State your method and explain why it works.
[1 \%]: Show your results.
\\
\textbf{Method: }\\
As shown in \cite{textbook}, we first find the solutions to the homogeneous equation, $v_1, v_2$, then construct the Green function by \begin{align*}
    \tilde{G}(x, s) = \begin{cases}
        v_1(s)v_2(x), &a \le s < x \\
        v_1(x)v_2(s), &x < s \le b
    \end{cases}
\end{align*}
and finally obtaining the Green function by $G(x, s) = \frac{\tilde{G}(x)}{Wp(x)} = \frac{\tilde{G}(x)}{W}$ to ensure that $C = 1$. 
\\
\textbf{Calculation: }\\
Notice that $v_1(x) = sin(x-a)$ and $v_2(x) = cos(x-b)$ are solutions to the homogeneous equation satisfying the boundary conditions.
\begin{align*}
    W &= v_1(x)v_2'(x) - v_1'(x)v_2(x) \\
    &= -cos(x-a)cos(x-b) - sin(x-a)sin(x-b) \\
    &= -cos(b-a)
\end{align*}
Note that if $cos(b-a) = 0$ we cannot divide by the wronskian, hence this is the difficulty we face.
Suppose that $cos(b-a) \neq 0$ we have \begin{align*}
    G(x, s) = \begin{cases}
        -\frac{v_1(s)v_2(x)}{cos(b-a)}, &a \le s < x \\
        -\frac{v_1(x)v_2(s)}{cos(b-a)}, &x < s \le b
    \end{cases}
\end{align*}

\vspace{20pt}
\textbf{II.}[2 \%] Write the generalized Legendre equation,
\begin{align*}
(1-x^2)\frac{d^2y}{dx^2}-2x\frac{dy}{dx}+\left\{n(n+1)-\frac{m^2}{1-\mu^2}\right\}y=0,
\end{align*}
as a Sturm-Liouville equation.\\
\textbf{Grading:}
[1 \%]: State your method and explain why it works.
[1 \%]: Show your results.
\vspace{20pt}
\\
\textbf{Method: }
The Sturm-Liouville equation is of the form $(p(x)y')' + q(x)y = -\lambda r(x)y$. We use our attention to notice the answer.\\
\textbf{Calculation:}\\
Notice that $(1-x^2)' = -2x$, hence our equation may be rearranged into \begin{align*}
    ((1-x^2) y')' = -(n(n+1) - \frac{m^2}{1-\mu^2})y
\end{align*}
\\
\textbf{III.}[2 \%] Show that
\begin{align*}
-(xy'(x))'=\lambda xy(x),
\end{align*}
is self-adjoint on the interval \((0,1)\), with \(x=0\) a singular
endpoint and \(x=1\) a regular endpoint with the condition \(y(1)=0\).\\
\textbf{Grading:}
[1 \%]: State your method and explain why it works.
[1 \%]: State your proof.
\\
\textbf{Method: }\\
For a linear operator $L$ to be self adjoint we must have $\langle Lf,g \rangle = \langle f,Lg \rangle$ for any $f, g$.
By \cite{textbook}, if $L$ is of the form $\frac{d}{dx}(p(x)\frac{d}{dx}) + q(x)$, it is self adjoint iff $[p(y_1y_2'-y_1'y_2)]_a^b = 0$.
\\
\textbf{Calculations:}\\
We let $L = -\frac{d}{dx}(x)$. Note that since for any $f, g \in C^2[0, 1]$ with $f(1) = g(1) = 0$, $[x(fg'-f'g)]_0^1 = 1(f(1)g'(1) - f'(1)g(1)) = 0$, we have $-L$ is self adjoint.
Now for any $f, g \in C^2[0,1]$, $\langle Lf, g\rangle = -\langle -Lf, g \rangle = -\langle f, -Lg \rangle = \langle f, Lg \rangle$, hence $L$ is self adjoint.


\bibliographystyle{plain}\
\bibliography{refs}
\end{document}
